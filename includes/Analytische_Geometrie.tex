\newpage
\part{Analytische Geometrie}

\section{Grundlagen}
\begin{itemize}

  \item \textbf{Vektoren aufstellen}:
  Sind $P(x_P~|~y_P)$ und $Q(x_Q~|~y_Q)$ zwei Punkte in $\mathbb{R}^2$
  bzw. $P(x_P~|~y_P~|~z_P)$ und $Q(x_Q~|~y_Q~|~z_Q)$ zwei Punkte in
  $\mathbb{R}^3$, so gilt:
  \begin{equation}
    \overrightarrow{PQ}=\left(
      \begin{array}{c}
        x_Q - x_P \\ y_Q - y_P \\ (z_Q - z_P)
      \end{array}
    \right)
  \end{equation}

  \item \textbf{Ortsvektor}:
  Der Ortsvektor eines Punktes $P(x~|~y)$ ist der Vektor vom
  Koordinatenursprung $O(0~|~0)$ zu diesem Punkt. Es gilt:
  \begin{equation}
    \overrightarrow{OP}=\left(
      \begin{array}{c} x \\ y \\ (z) \end{array}
    \right)
  \end{equation}

  \item \textbf{Betrag}:
  Ist der Betrag des Vektors gleich 1, so spricht man von einem
  Einheitsvektor. Für den Betrag eines Vektors gilt:
  \begin{equation}
    \overrightarrow{a}=
      \left(\begin{array}{c} x \\ y \\ (z) \end{array}\right)
      \Rightarrow \left| \overrightarrow{a} \right|
      = \sqrt{ x^2 + y^2 (+ z^2) }
  \end{equation}

  \item \textbf{Addition}:
  Die Summe zweier Vektoren kann beschrieben werden durch:
  \begin{equation}
    (\overrightarrow{a} = \overrightarrow{AB}) \wedge
    (\overrightarrow{b} = \overrightarrow{BC}) \Rightarrow \overrightarrow{c} = \overrightarrow{a} + \overrightarrow{b} = \overrightarrow{AC}
  \end{equation}

\end{itemize}

\section{Linearkombination von Vektoren}
\begin{itemize}
  \item \textbf{kollinear}:
  Ein Vektor ist ein Vielfaches des anderen. $\overrightarrow{a}$
  und $\overrightarrow{b}$ sind linear abhängig und parallel
  zueinander; sie spannen keine Ebene auf.
  \begin{equation}
    \overrightarrow{a} = r \cdot \overrightarrow{b}
  \end{equation}

  \item \textbf{komplanar}:
  Mit zwei linear unabhängigen Vektoren lässt sich ein dritter
  darstellen. $\overrightarrow{a}$, $\overrightarrow{b}$ und
  $\overrightarrow{c}$ liegen in einer Ebene und spannen keinen
  Raum auf.
  \begin{equation}
    \overrightarrow{a} = r \cdot \overrightarrow{b} + s \cdot \overrightarrow{c}
  \end{equation}
\end{itemize}

\newpage
\section{Skalarprodukt}
Für zwei Vektoren
$\overrightarrow{a} = \left(\begin{array}{c} x_a \\ y_a \\ (z_a)
\end{array}\right)$ und $\overrightarrow{b} = \left(\begin{array}{c}
x_b \\ y_b \\(z_b) \end{array}\right)$ gilt:
\begin{align}
  \overrightarrow{a} \cdot \overrightarrow{b}
  &= x_a \cdot x_b + y_a \cdot y_b (+z_a \cdot z_b) \\
  \cos \measuredangle (\overrightarrow{a},\overrightarrow{b})
  &= \frac{\overrightarrow{a} \cdot \overrightarrow{b}}
  {| \overrightarrow{a} | \cdot | \overrightarrow{b} |}
\end{align}
Mit dem CAS berechnet man das Skalarprodukt durch
\glqq dotP($\overrightarrow{a}, \overrightarrow{b}$)\grqq.
% TODO: Veranschaulichung ergänzen.


\section{Vektorprodukt}
Für zwei Vektoren
$\overrightarrow{a} = \left(\begin{array}{c} x_a \\ y_a \\ (z_a) \end{array} \right)$
und
$\overrightarrow{b} = \left(\begin{array}{c} x_b \\ y_b \\(z_b) \end{array}\right)$
gilt:
\begin{align}
  \overrightarrow{a} \times \overrightarrow{b}
  &= - \overrightarrow{b} \times \overrightarrow{a}
  = \left( \begin{array}{c}
    y_a \cdot z_b - z_a \cdot y_b \\
    z_a \cdot x_b - x_a \cdot z_b \\
    x_a \cdot y_b - y_a \cdot x_b
  \end{array} \right) \\
  \sin \measuredangle (\overrightarrow{a}, \overrightarrow{b})
  &= \frac{| \overrightarrow{a} \times \overrightarrow{b} |}
  {| \overrightarrow{a} | \cdot | \overrightarrow{b} |}
\end{align}
Mit dem CAS berechnet man das Vektorprodukt durch
\glqq crossP($\overrightarrow{a}, \overrightarrow{b}$)\grqq.
% TODO: Veranschaulichung ergänzen
