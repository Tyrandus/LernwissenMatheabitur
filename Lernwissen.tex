% Format und Encodierung
\documentclass[a4paper, pdftex, 12pt, ngerman]{article}
\usepackage[utf8]{inputenc}
\usepackage[T1]{fontenc}
\usepackage{babel}

% Links
\usepackage{hyperref}
\usepackage{url}

% Grafiken
\usepackage{graphicx}
\usepackage{float}

% Mathematik
\usepackage{amsmath}
\usepackage{amssymb}
\usepackage{amsthm}
\usepackage{mathtools}

% Listings
\usepackage{verbatim}
\usepackage{listings}
\usepackage{color}

% Tabellen
\usepackage{longtable}

% Kopf- und Fußzeile
\usepackage[headsepline,footsepline]{scrpage2}

% Seitenränder
\usepackage[margin=1in]{geometry}

\begin{document}

% Kopf- und Fußzeilendefinition
\pagestyle{scrheadings}
\clearscrheadfoot
\cfoot{Kurzwissen für das Mathematikabitur in Thüringen (\today)}
\ofoot{\pagemark}
\setheadtopline{0pt}
\setheadsepline{0pt}

\part{Analysis I}
\section{Funktionsuntersuchung/Kurvendiskussion}
\begin{table}[H]
  \centering
  \begin{tabular}{l|l}

    (1) Definitionsbereich $\mathbb{D}$
    & alle x, für die ein Funktionswert definiert ist \\
    & $f(x)=\sin(x) \Rightarrow \mathbb{D}: \mathbb{R}$ \\
    & $f(x)=\ln(x) \Rightarrow \mathbb{D}: \mathbb{R}_0^+ $ \\
    \\ \hline \\

    (2) Wertebereich $\mathbb{W}$
    & alle y, denen ein Urbild x zugeordnet werden kann \\
    & $f(x) = \sin(x) \Rightarrow \mathbb{W} = \{y\in\mathbb{R}~|~-1 \le y \le 1\}$ \\ \\ \hline \\

    (3) Achsenschnittpunkte
    & $f(x_0)=0, f(0)=y_0$ \\ \\ \hline \\

    (4) Grenzwerte
    & $f(x) = x^3 \Rightarrow \lim\limits_{x \rightarrow \infty}f(x) = \infty$ \\ \\ \hline \\

%   \end{tabular}
% \end{table}


% \begin{table}
%   \begin{tabular}{l|l}

    (5) Symmetrie
    & axialsymmetrisch: $f(x)=f(-x)$ \\
    & oder*: alle Exponenten sind gerade \\ \\
    & punktsymmetrisch: $f(-x)=-f(x)$ \\
    & oder*: alle Exponenten sind ungerade \\
    & (*gilt nur für ganz-rationale Funktionen) \\ \\ \hline \\

    (6) Extrema
    & Bedingung 1: $f'(x_0)=0$ \\
    & Bedingung 2: $f''(x_0) \neq 0$ \\ \\
    & $f''(x)=0 \Rightarrow $ Sattelpunkt \\
    & $f''(x)>0 \Rightarrow $ Tiefpunkt $(x_0 ~|~ f(x_0))$ \\
    & $f''(x)<0 \Rightarrow $ Hochpunkt $(x_0 ~|~ f(x_0))$ \\ \\ \hline \\

    (7) Wendepunkte
    & Bedingung 1: $f''(x_0)=0$ \\
    & Bedingung 2: $f'''(x_0) \neq 0$ \\ \\ \hline \\

    (8) Graph der Funktion
    & Achsenbeschriftung und -einteilung beachten

  \end{tabular}
\end{table}

\section{Differentiation}
\subsection{Ableitungsregeln}
\renewcommand{\arraystretch}{1.5}
\begin{table}[H]
  % \centering
  \begin{tabular}{lll}
    (1) Potenzregel     &$f(x)=x^n$           &$f'(x)=nx^{n-1}$ \\
    (2) Summenregel     &$f(x)=u(x)\pm v(x)$  &$f'(x)=u'(x) \pm v'(x)$ \\
    (3) Produktregel    &$f(x)=u(x)\cdot v(x)$     &$f'(x)=u'(x)\cdot v(x) + v'(x)\cdot u(x)$ \\
    (4) Quotientenregel &$f(x)=\frac{u(x)}{v(x)}$     &$f'(x)=\frac{u'(x)\cdot v(x)-v'(x)\cdot u(x)}{v^2}(=\frac{u'v-v'u}{v^2})$\\
    (5) Kettenregel     &$f(x)=u(v(x))$       &$f'(x)=u'(v(x))\cdot v'(x)$
  \end{tabular}
\end{table}
\renewcommand{\arraystretch}{1}

% TODO: Beste Repräsentation festlegen
% \begin{align}
%   f(x)&=x^n            & f'(x)&=nx^{n-1} \\
%   f(x)&=u(x)\pm v(x)   & f'(x)&=u'(x) \pm v'(x) \\
%   f(x)&=u(x)\cdot v(x)      & f'(x)&=u'(x)\cdot v(x) + v'(x)\cdot u(x) \\
%   f(x)&=\frac{u(x)}{v(x)}     &f'(x)&=\frac{u'(x)\cdot v(x)-v'(x)\cdot u(x)}{v^2}(=\frac{u'v-v'u}{v^2})\\
%   f(x)&=u(v(x))        & f'(x)&=u'(v(x))\cdot v'(x)
% \end{align}
% Die oben stehenden Regeln sind benannt als Potenzregel (1), Summenregel (2), Produktregel (3), Quotientenregel (4) und Kettenregel (5).

%TODO: Beispiele an anderer Stelle anordnen
% \begin{center}
%   \begin{table}[H]
%     \centering
%     \begin{tabular}{ll}
%       $f(x)=x^2+2$                      &$f'(x)=2x$ \\
%       $f(x)=\sqrt{x}=x^{\frac{1}{2}}$   &$f'(x)=\frac{1}{2}x^{-\frac{1}{2}}=\frac{1}{2\sqrt{x}}$ \\
%       $f(x)=\sin(x)$                    &$f'(x)=\cos(x)$ \\
%       $f(x)=\cos(x)$                    &$f'(x)=-\sin(x)$ \\
%       $f(x)=\frac{1}{x}=x^{-1}$         &$f'(x)=-x^{-2}=-\frac{1}{x^2}$ \\
%     \end{tabular}
%   \end{table}
% \end{center}

\subsection{Tangenten und Normalen}
\begin{table}[H]
  \begin{tabular}{ll}
    Tangente: &$y-y_0=m_T\cdot (x-x_0)$ für $P_0(x_0, y_0)$ und $m_T=f'(x)$ \\
    Normale:  &$y-y_0=m_N\cdot (x-x_0)$ für $P_0(x_0, y_0)$ und $m_N=-\frac{1}{m_T}$ \\
    & steht immer orthogonal (senkrecht) zur Tangente
  \end{tabular}
\end{table}

\subsection{Berührungspunkt}
\begin{itemize}
  \item Haben zwei Funktionen $f(x)$ und $g(x)$ an einer Stelle $x_a$ denselben Funktionswert $f(x_a)=g(x_a)$ und denselben Anstieg $f'(x_a)=g'(x_a)$, so berühren sie sich im Punkt $(x_a~|~f(x_a))$.
\end{itemize}

\subsection{Gebrochen rationale Funktionen}
\begin{itemize}
  \item Eine Funktion gilt als \textbf{echt gebrochen}, wenn der höchste Grad aller Potenzen der Funktionsvariablen im Zähler kleiner als der höchste Grad aller Potenzen der Funktionsvariablen im Nenner ist.\\
  \textit{Beispiel: $f(x)=\frac{2x^2}{x^3}$ ist eine echt gebrochen rationale Funktion, $f(x)=\frac{2x^3}{x^2}$ nicht.}

  \item Hat eine Funktion $f$ die Form $f(x)=\frac{ax^k}{bx^k}$, so gilt: $\lim\limits_{x \rightarrow \infty}f(x)=\frac{a}{b}$ (\textbf{Limestrick})

  \item Eine \textbf{Polstelle} hat eine gebrochen rationale Funktion $f(x)=\frac{u(x)}{v(x)}$ an der Stelle $x_P$, wenn $v(x_P)=0$ und $u(x_P)\neq0$.

  \item Eine \textbf{Definitionslücke} hat eine gebrochen rationale Funktion $f(x)=\frac{u(x)}{v(x)}$ an der Stelle $x_P$, wenn $v(x_P)=u(x_P)=0$.

  \item \textbf{Asymptoten}: Alle Polstellen sind x-Asymptoten. y-Asymptoten entsprechen dem ganzrationalen Teil am Ende der Polynomdivision.

  \item Auf dem CAS lässt sich \textbf{Polynomdivision} über den Befehl \glqq propFrac\grqq~ausführen. Es ist allerdings empfohlen, die entsprechenden Schritte auch von Hand vollziehen zu können.
\end{itemize}

\newpage
\section{Integration}
\subsection{Regeln der Integralrechnung}
Gleichung (1) ist bekannt als die Newton-Leibniz-Formel.

\begin{align}
  \int\limits_a^bf(x)dx &= F(b)-F(a) \\
  \int\limits_a^bf(x)dx &= -\int\limits_b^af(x)dx \\
  \int\limits_a^af(x)dx &= 0 \\
  \int\limits_a^cf(x)dx &= \int\limits_a^bf(x)dx + \int\limits_b^cf(x)dx \\
  \int\limits_a^bk\cdot f(x)dx &= k\cdot \int\limits_a^bf(x)dx\\
  \int\limits_a^bf(x)dx + \int\limits_a^bg(x)dx &= \int\limits_a^b(f(x)+g(x))dx
\end{align}

% TODO: Beispiele für Integration abtippen und an angemessener Stelle einbinden

\subsection{Unbestimmtes Integral}
Ein unbestimmtes Integral bezeichnet die Gesamtheit aller Stammfunktionen einer Funktion, da es keine Integrationsgrenzen enthält.
\begin{equation}
  \int f(x)dx=F(x)+C ~|~ C\in\mathbb{R}
\end{equation}

\subsection{Uneigentliches Integral}
Ein uneigentliches Integral ist ein Integral, bei dem mindestens eine der Integrationsgrenzen $\pm\infty$ ist. Die Berechnung erfolgt dann über die Grenzwertbetrachtung der Stammfunktion.
\begin{equation}
  \int\limits_F^\infty f(x) = \lim\limits_{n\rightarrow\infty}(F(n)-F(1))
\end{equation}
% TODO: Beispiel einfügen / referenzieren
% \begin{align*}
%   f(x) &= \frac{1}{x^{2}} \\
%   \int\limits_1^nf(x)dx &= -\frac{1}{x} = -\frac{1}{n}+1 \\
%   \lim\limits_{n\rightarrow\infty}(-\frac{1}{n}+1) &= 1 \Rightarrow \int\limits_1^nf(x)dx = -\frac{1}{x} = 1
% \end{align*}
\newpage
\part{Analytische Geometrie}

\section{Grundlagen}
\begin{itemize}

  \item \textbf{Vektoren aufstellen}:
  Sind $P(x_P~|~y_P)$ und $Q(x_Q~|~y_Q)$ zwei Punkte in $\mathbb{R}^2$
  bzw. $P(x_P~|~y_P~|~z_P)$ und $Q(x_Q~|~y_Q~|~z_Q)$ zwei Punkte in
  $\mathbb{R}^3$, so gilt:
  \begin{equation}
    \overrightarrow{PQ}=\left(
      \begin{array}{c}
        x_Q - x_P \\ y_Q - y_P \\ (z_Q - z_P)
      \end{array}
    \right)
  \end{equation}

  \item \textbf{Ortsvektor}:
  Der Ortsvektor eines Punktes $P(x~|~y)$ ist der Vektor vom
  Koordinatenursprung $O(0~|~0)$ zu diesem Punkt. Es gilt:
  \begin{equation}
    \overrightarrow{OP}=\left(
      \begin{array}{c} x \\ y \\ (z) \end{array}
    \right)
  \end{equation}

  \item \textbf{Betrag}:
  Ist der Betrag des Vektors gleich 1, so spricht man von einem
  Einheitsvektor. Für den Betrag eines Vektors gilt:
  \begin{equation}
    \overrightarrow{a}=
      \left(\begin{array}{c} x \\ y \\ (z) \end{array}\right)
      \Rightarrow \left| \overrightarrow{a} \right|
      = \sqrt{ x^2 + y^2 (+ z^2) }
  \end{equation}

  \item \textbf{Addition}:
  Die Summe zweier Vektoren kann beschrieben werden durch:
  \begin{equation}
    (\overrightarrow{a} = \overrightarrow{AB}) \wedge
    (\overrightarrow{b} = \overrightarrow{BC}) \Rightarrow \overrightarrow{c} = \overrightarrow{a} + \overrightarrow{b} = \overrightarrow{AC}
  \end{equation}

\end{itemize}

\section{Linearkombination von Vektoren}
\begin{itemize}
  \item \textbf{kollinear}:
  Ein Vektor ist ein Vielfaches des anderen. $\overrightarrow{a}$
  und $\overrightarrow{b}$ sind linear abhängig und parallel
  zueinander; sie spannen keine Ebene auf.
  \begin{equation}
    \overrightarrow{a} = r \cdot \overrightarrow{b}
  \end{equation}

  \item \textbf{komplanar}:
  Mit zwei linear unabhängigen Vektoren lässt sich ein dritter
  darstellen. $\overrightarrow{a}$, $\overrightarrow{b}$ und
  $\overrightarrow{c}$ liegen in einer Ebene und spannen keinen
  Raum auf.
  \begin{equation}
    \overrightarrow{a} = r \cdot \overrightarrow{b} + s \cdot \overrightarrow{c}
  \end{equation}
\end{itemize}

\newpage
\section{Skalarprodukt}
Für zwei Vektoren
$\overrightarrow{a} = \left(\begin{array}{c} x_a \\ y_a \\ (z_a)
\end{array}\right)$ und $\overrightarrow{b} = \left(\begin{array}{c}
x_b \\ y_b \\(z_b) \end{array}\right)$ gilt:
\begin{align}
  \overrightarrow{a} \cdot \overrightarrow{b}
  &= x_a \cdot x_b + y_a \cdot y_b (+z_a \cdot z_b) \\
  \cos \measuredangle (\overrightarrow{a},\overrightarrow{b})
  &= \frac{\overrightarrow{a} \cdot \overrightarrow{b}}
  {| \overrightarrow{a} | \cdot | \overrightarrow{b} |}
\end{align}
Mit dem CAS berechnet man das Skalarprodukt durch
\glqq dotP($\overrightarrow{a}, \overrightarrow{b}$)\grqq.
% TODO: Veranschaulichung ergänzen.


\section{Vektorprodukt}
Für zwei Vektoren
$\overrightarrow{a} = \left(\begin{array}{c} x_a \\ y_a \\ (z_a) \end{array} \right)$
und
$\overrightarrow{b} = \left(\begin{array}{c} x_b \\ y_b \\(z_b) \end{array}\right)$
gilt:
\begin{align}
  \overrightarrow{a} \times \overrightarrow{b}
  &= - \overrightarrow{b} \times \overrightarrow{a}
  = \left( \begin{array}{c}
    y_a \cdot z_b - z_a \cdot y_b \\
    z_a \cdot x_b - x_a \cdot z_b \\
    x_a \cdot y_b - y_a \cdot x_b
  \end{array} \right) \\
  \sin \measuredangle (\overrightarrow{a}, \overrightarrow{b})
  &= \frac{| \overrightarrow{a} \times \overrightarrow{b} |}
  {| \overrightarrow{a} | \cdot | \overrightarrow{b} |}
\end{align}
Mit dem CAS berechnet man das Vektorprodukt durch
\glqq crossP($\overrightarrow{a}, \overrightarrow{b}$)\grqq.
% TODO: Veranschaulichung ergänzen


\end{document}
